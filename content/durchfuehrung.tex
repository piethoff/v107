\section{Durchführung}
\label{sec:Durchführung}
\subsection{Vermessung der Kugeln}
In dem Versuch werden zwei verschieden große und schwere Kugeln verwendet.
Der Durchmesser der beiden wird mit einer Schieblehre gemessen und das Gewicht dieser mit einer Waage.
Die beiden aufgeführten Messungen werden drei mal wiederholt um eine genauere bestimmung des Realwerts zu ermöglichen.
\subsection{Bestimmung der Viskosität von Wasser}
Es wird die Temperaturabhängigkeit von Wasser untersucht.
Zunächst wird die Viskosität des Wassers bei Raumtemperatur bestimmt. Dies muss mit der kleinen Kugel erfolgen,
da die Apperaturkonstante für die große Kugel noch nicht bekannt ist. Es dabei darauf zu achten, dass sich keine Luftbläschen bei der Befüllung des Fallrohrs an der Rohrwand bilden.
Dafür wird die kleine Kugel fallen gelassen.
Die Messung der Fallzeit beginnt sobald diese die rote Markierung passiert und endet wenn diese die letzte Markierung überschreitet.
Die Zeiten werden mit zwei Stoppuhren genommen.
Der Vorgang wird fünf mal wiederholt um eine höhere Genauigkeit zu erreichen.
\subsection{Bestimmung der Apperaturkonstante für die große Kugel}
Anschließend wird die Messung mit der großen Kugel
wiederholt, sodass mit der zuvor ermittelten Viskosität auf die Apperaturkonstante geschlossen werden kann.
Auch hier muss wieder darauf geachtet werden, dass durch den Wechsel der Kugeln keine Luftbläschen an der Rohrinnenwand entstehen.
\subsection{Untersuchung der Temperaturabhängigkeit der Viskosität}
Die Messung mit der großen Kugel wird erneut wiederholt, jedoch wird nun die Temperatur über ein Wärmebad verändert.
Es werden Werte für zehn verschiedene Temperaturen gemessen.
Aus den so erhaltenen Messpunkten werden schließlich die Konstanten der Andradeschen Gleichung \eqref{eq:and} ermittelt
und somit auch die Temperaturabhängigkeit von Wasser.
