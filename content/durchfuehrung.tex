\section{Durchführung}
\label{sec:Durchführung}
Es wird die Temperaturabhängigkeit von Wasser untersucht.
Es wird zunächst die Viskosität des Wassers bei Raumtemperatur bestimmt. Dies muss mit der kleinen Kugel erfolgen, 
da die Apperaturkonstante für die große Kugel noch nicht bekannt ist. Anschließend wird die Messung mit der großen Kugel
wiederholt, sodass mit der zuvor ermittelten Viskosität auf die Apperaturkonstante geschlossen werden kann.
Die Messung mit der großen Kugel wird erneut wiederholt, jedoch wird nun die Temperatur über ein Wärmebad verändert.
Aus den so erhaltenen Messpunkten werden schließlich die Konstanten der Andradeschen Gleichung \eqref{eq:dubbi} ermittelt 
und somit auch die Temperaturabhängigkeit von Wasser.
