\section{Durchführung}
\label{sec:Durchführung}
Es wird die Temperaturabhängigkeit von Wasser untersucht.
Zunächst wird die Viskosität des Wassers bei Raumtemperatur bestimmt. Dies muss mit der kleinen Kugel erfolgen,
da die Apperaturkonstante für die große Kugel noch nicht bekannt ist. Dafür wird die kleine Kugel fallen gelassen.
Die Messung der Fallzeit beginnt sobald diese die rote Markierung passiert und endet wenn diese die letzte Markierung überschreitet.
Die Zeiten werden mit zwei Stoppuhren genommen.
Der Vorgang wird fünf mal wiederholt um eine höhere Genauigkeit zu erreichen.
Anschließend wird die Messung mit der großen Kugel
wiederholt, sodass mit der zuvor ermittelten Viskosität auf die Apperaturkonstante geschlossen werden kann.
Die Messung mit der großen Kugel wird erneut wiederholt, jedoch wird nun die Temperatur über ein Wärmebad verändert.
Es werden Werte für zehn verschiedene Temperaturen gemessen.
Aus den so erhaltenen Messpunkten werden schließlich die Konstanten der Andradeschen Gleichung \eqref{eq:and} ermittelt
und somit auch die Temperaturabhängigkeit von Wasser.
