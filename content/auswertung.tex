\section{Auswertung}
\label{sec:Auswertung}
\subsection{Bestimmung der Viskosität von destilliertem Wasser bei konstanter Temperatur}
Die Dichte der kleinen Kugel erhält man aus den gemessenen Werten für die Masse und den Durchmesser, welche in Tabelle \ref{tab:kl_maße} aufgetragen sind.
\begin{table}[H]
  \caption{Gemessene Werte für die Größe und das Gewicht der kleinen Kugel.}
  \label{tab:kl_maße}
  \centering
  \sisetup{table-format=2.2}
  \begin{tabular}{cSSS}
    \toprule
    \midrule
    {$m/\si{\gram}$} & 4.45 & 4.44 & 4.46 \\
    {$D/\si{\milli\meter}$} & 15.59 & 15.60 & 15.59 \\
    \bottomrule
  \end{tabular}
\end{table}
\noindent Im Mittel erhält man dann für den Durchmesser und die Masse:
\begin{equation*}
  \bar{m}=\SI{4.45\pm 0.01}{\gram}
\end{equation*}
\begin{equation*}
  \bar{D}=\SI{15.59\pm 0.004 }{\milli\meter}
\end{equation*}
\noindent Die Ungenauigkeit errechnet sich hier über Streung des Mittelwertes:
 \begin{equation}
   \label{eq:streu}
   \Delta \bar{x} =\sqrt{\frac{1}{N(N-1)}\sum_{i=1}^N(x_i-\bar{x})^2}
 \end{equation}
 Die Kugel hat ein Volumen von:
 \begin{equation}
   V = \frac{4}{3} \pi r^3 = \SI{1.984\pm0.002e-6}{\cubic\meter}
 \end{equation}
 Die Ungenauigkeit des Volumens berechnet sich über die Gaußsche Fehlerfortpflanzung.
 Damit ergibt sich für die Kugel eine Dichte von:
 \begin{equation}
   \rho_K =\frac{\bar{m}} {V}=\SI{2242.94\pm5.33}{\kilo\gram\per\cubic\meter}
 \end{equation}
Die Unsicherheit errechnet sich hier über die Gaußsche Fehlerfortpflanzung.
Die gemessenen Werte für die Fallzeit der Kugel sind in Tabelle \ref{tab:kl_fall} eingetragen.
\begin{table}[H]
    \centering
    \caption{Fallzeiten der kleinen Kugel.}
    \label{tab:kl_fall}
    \begin{tabular}{S[table-format=2.2] S[table-format=2.2] }
        \toprule
        {$1.Messung/\si{\second}$} & {$2.Messung/\si{\second}$} \\
        \midrule
        12.22   & 12.29 \\
        12.22   & 12.07 \\
        12.16   & 12.18 \\
        12.19   & 12.07 \\
        12.22   & 12.13 \\
        \bottomrule
    \end{tabular}
\end{table}
\noindent 
Daraus ergibt sich im Mittel folgende Fallzeit:
\begin{equation*}
  \label{eq:fallzeit}
  \bar{t}=\SI{12.175\pm0.022}{\second}
\end{equation*}
Die Ungenauigkeit errechnet sich hier über die Streuung des Mittelwertes nach Gleichung \eqref{eq:streu}.
Die Apparaturkonstante ist in der Anleitung für die Kleine Kugel mit
\mbox{$K_\text{kl}=\SI{0.0764e-6}{\pascal\cubic\meter\per\kilo\gram}$}\cite{v107}
angegeben.
Die Dichte des Wassers wird als eine Konstante mit $\rho_\text{Fl} = 988\si{\kilo\gram\per\cubic\meter}$\cite{dWasser}.
Damit folgt für die Viskosität von destilliertem Wasser folgender Wert:
\begin{equation*}
  \label{eq:vis}
  \upeta = \SI{1.167\pm0.005e-3}{\newton\second\per\meter\squared}
\end{equation*}
Die Ungenauigkeit errechnet sich hier mit der Gaußschen Fehlerfortpflanzung nach:
\begin{equation*}
  \Delta\upeta = \sqrt{(\Delta t K(\rho_K - \rho_\text{Fl})^2 +(Kt\symup{\Delta}\rho_K)^2}
\end{equation*}
\subsection{Bestimmung der Apparaturkonstante für die große Kugel}
Die Dichte der großen Kugel erhält man aus den gemessenen Werten für die Masse und den Durchmesser, welche in Tabelle \ref{tab:g_maße} aufgetragen sind.
\begin{table}[H]
  \caption{Gemessene Werte für die Größe und das Gewicht der großen Kugel.}
  \label{tab:g_maße}
  \centering
  \sisetup{table-format=2.2}
  \begin{tabular}{cSSS}
    \toprule
    \midrule
    {$m/\si{\gram}$} & 4.95 & 4.95 & 4.94 \\
    {$D/\si{\milli\meter}$} & 15.76 & 15.76 & 15.76 \\
    \bottomrule
  \end{tabular}
\end{table}
\noindent Im Mittel erhält man dann für den Durchmesser und die Masse:
\begin{equation*}
  \bar{m}=\SI{4.947\pm0.003}{\gram}
\end{equation*}
\begin{equation*}
  \bar{D}=\SI{15.76}{\milli\meter}
\end{equation*}
\noindent Die Ungenauigkeit errechnet sich hier über Streuung des Mittelwertes nach Gleichung\eqref{eq:streu}.
Die Kugel hat ein Volumen von
\begin{equation*}
  V =\SI{2.049e-6}{\cubic\meter}
\end{equation*}
und damit eine Dichte von
\begin{equation*}
  \rho_K = \SI{2413.49\pm1.46}{\kilo\gram\per\cubic\meter}.
\end{equation*}
Die Ungenauigkeit errechnet sich über die Gaußsche Fehlerfortpflanzung:
\begin{equation*}
  \Delta \rho_K = \Delta m \frac{1}{V}
\end{equation*}
Die gemessenen Werte für die Fallzeit befinden sich in Tabelle \ref{tab:fall_g}.
\begin{table}[H]
    \centering
    \caption{Fallzeiten der großen Kugel.}
    \label{tab:fall_g}
    \begin{tabular}{S[table-format=2.2] S[table-format=2.2] }
        \toprule
        {$1.Messung/\si{\second}$} & {$2.Messung/\si{\second}$} \\
        \midrule
        70.34   & 70.13 \\
        69.56   & 69.76 \\
        70.22   & 69.46 \\
        69.22   & 69.34 \\
        69.84   & 69.69 \\
        \bottomrule
    \end{tabular}
\end{table}
\noindent
 Im Mittel errechnet sich daraus ein Wert von
 \begin{equation*}
   \bar{t} = \SI{69.76\pm0.12}{\second}
 \end{equation*}
 Die Ungenauigkeit errechnet sich analog zu der Fallzeit der kleinen Kugel \ref{eq:fallzeit}.
 Mit der vorher errechneten Viskosität $\upeta=\SI{1.167e-3}{\newton\second\per\meter\squared}$, lässt sich die Apparaturkonstante für die große Kugel über
 \begin{equation*}
   K =\frac{\upeta}{(\rho_K - \rho_\text{Fl})\cdot \bar{t}}
 \end{equation*}
 bestimmen.
 Damit ergibt sich eine Apparaturkonstante von:
 \begin{equation*}
   K=\SI{1.1735\pm 0.0056e-8}{\pascal\cubic\meter\per\kilo\gram}
 \end{equation*}
 Die Ungenauigkeit errechnet sich über die Gaußsche Fehlerfortpflanzung nach:
 \begin{equation*}
   \Delta K = \sqrt{(\Delta t \frac{- \upeta}{(\rho_K - \rho_\text{Fl})t^2})^2 +(\Delta \upeta \frac{1}{(\rho_K - \rho_\text{Fl})t})^2 + (\Delta \rho_K \frac{-\upeta}{({\rho_K} - \rho_\text{Fl})^2t})^2}
 \end{equation*}
%
\subsection{Temperaturabhängigkeit der Viskosität}
Es wurden folgende Messwerte aufgenommen:
\begin{table}[H]
    \centering
    \caption{Messwerte zur Bestimmung der temperaturabhängigen Viskosität.}
    \begin{tabular}{S[table-format=3.2] S[table-format=2.2(3)] S[table-format=1.3(4)] S[table-format=1.2]}
    \toprule
    {Temperatur$/\si{\celsius}$} & {Fallzeit$/\si{\second}$} & {$\eta /\si{\milli\pascal\second}$} & {$\frac{1}{T} /\SI[per-mode=reciprocal]{e-3}{\per\kelvin}$}\\
    \midrule
    305.15  & 53.17\pm0.25  & 0.889\pm0.006     & 3.28\\
    307.15  & 51.18\pm0.14  & 0.856\pm0.005     & 3.26\\
    311.15  & 46.84\pm0.89  & 0.783\pm0.002     & 3.21\\
    318.15  & 42.73\pm0.11  & 0.715\pm0.004     & 3.14\\
    321.15  & 40.27\pm0.04  & 0.674\pm0.003     & 3.11\\
    325.15  & 38.07\pm0.16  & 0.637\pm0.004     & 3.08\\
    328.15  & 36.84\pm0.34  & 0.616\pm0.006     & 3.05\\
    333.15  & 34.08\pm0.21  & 0.570\pm0.004     & 3.00\\
    337.15  & 31.96\pm0.66  & 0.534\pm0.011     & 2.97\\
    343.15  & 30.29\pm0.09  & 0.507\pm0.002     & 2.91\\
    \bottomrule
    \end{tabular}
\end{table}
\noindent 
Es wurden zusätzlich die errechneten Werte für $\eta$ mit
\begin{align}
    \eta &= K(\rho_K-\rho_{Fl})t \\
    \symup{\Delta}\eta &= \sqrt{\left(Kt\symup{\Delta}\rho_K\right)^2+\left(K(\rho_K-\rho_{Fl})\symup{\Delta}t\right)^2}
\end{align}
und $1/T$ aufgetragen.
Die Werte werden nun nach Gleichung \eqref{eq:and} mit Python/SciPy gefittet.
Messwerte und Curve-Fit sind in Abbildung \ref{fig:temp} abgebildet.
Es wurde die Funktion
\begin{equation}
    \eta (T) = A \exp{\frac{B}{T}}
\end{equation}
für die Regression verwendet.
Die Fehlerbalken resultieren aus den fehlerbehafteten Werten für $\eta$.
\begin{figure}[H]
    \centering
    \includegraphics{build/temp.pdf}
    \caption{Messwerte und Regression für die temperaturabhängige Viskosität.}
    \label{fig:temp}
\end{figure}
\noindent Die Parameter sind hierbei:
\begin{align}
    A &= \SI{5.32\pm0.31e-6}{\pascal\second} \\
    B &= \SI{1.56\pm0.02e3}{\kelvin}
\end{align}
Somit ist die Temperaturabhängigkeit der Viskosität bestimmt.
%
\subsection{Berechnung der Reynoldschen Zahl}
Die Reynolds-Zahl wird mithilfe von Gleichung \eqref{eqn:rey} für Raumtemperatur bestimmt.
Hierbei ist die Dichte des Wassers
\mbox{$\rho_\text{Fl} = \SI[per-mode=reciprocal]{988}{\kg\per\meter\cubed}$},
die zuvor berechnete Viskosität bei Raumtemperatur
\mbox{$\eta = \SI[per-mode=reciprocal]{1.167\pm0.005}{\newton\second\per\meter\squared}$} und
$v=x/t$ mit $x=\SI{100}{\milli\meter}$.
Für die große Kugel ergibt sich mit der Falldauer
\mbox{$t = \SI[per-mode=reciprocal]{69.76\pm0.12}{\second}$} und
dem Durchmesser der Kugel
\mbox{$D = \SI[per-mode=reciprocal]{15.76}{\milli\meter}$}
eine Reynolds-Zahl von
\begin{equation}
    \mathit{R\kern-.04em e} = \num{19.13\pm0.09e-3}.
\end{equation}
%
Für die kleine Kugel ergibt sich mit
der Falldauer
\mbox{$t = \SI[per-mode=reciprocal]{12.175\pm0.022}{\second}$} und
dem Durchmesser der Kugel
\mbox{$D = \SI[per-mode=reciprocal]{15.590\pm0.004}{\milli\meter}$}
eine Reynolds-Zahl von
\begin{equation}
    \mathit{R\kern-.04em e} = \num{108.41\pm0.03e-3}.
\end{equation}
%
Die Fehler berechnet sich nach:
\begin{equation}
    \symup{\Delta}\mathit{R\kern-.04em e} = \sqrt{
        \left(\rho_\text{Fl} \frac{xD}{\eta t^2}\symup{\Delta}t \right)^2 +
        \left(\rho_\text{Fl} \frac{x}{\eta t}\symup{\Delta}D \right)^2 +
        \left(\rho_\text{Fl} \frac{xD}{\eta^2 t}\symup{\Delta}\eta \right)^2
    }
\end{equation}
