\section{Auswertung}
\label{sec:Auswertung}
\subsection{Bestimmung der Viskosität von destilliertem Wasser bei konstanter Temperatur}
Die Viskosität von destilliertem Wasser bei konstanter Temperatur lässt sich über die Gleichung\eqref{eq:dicht} berechnen.
Die Dichte der Kugel erhält man aus den gemessenen Werten für die Masse und den Durchmesser, welche in Tabelle aufgetragen sind.
\begin{table}[H]
  \caption{Ergebnisse der Messung.}
  \label{tab:kl_maße}
  \centering
  \sisetup{table-format=2.2}
  \begin{tabular}{cSSS}
    \toprule
    \midrule
    {$m/\si{\gram}$} & 4.45 & 4.44 & 4.46 \\
    {$D/\si{\milli\meter}$} & 15.59 & 15.60 & 15.59 \\
    \bottomrule
  \end{tabular}
\end{table}
\noindent Im Mittel erhält man dann für den Durchmesser und die Masse:
\begin{equation*}
  \bar{m}=\SI{4.45\pm 0.01}{\gram}
\end{equation*}
\begin{equation*}
  \bar{D}=\SI{15.59\pm 0.14 }{\milli\meter}
\end{equation*}
\noindent Die Ungenauigkeit errechnet sich hier über Streung des Mittelwertes:
 \begin{equation}
   \Delta \bar{x} =\sqrt{\frac{1}{N(N-1)}\sum_{i=1}^N(x_i-\bar{x})^2}
 \end{equation}
\begin{table}[H]
    \centering
    \caption{Fallzeiten der kleinen Kugel.}
    \label{tab:kl_fall}
    \begin{tabular}{S[table-format=2.2] S[table-format=2.2] }
        \toprule
        {$1.Messung/\si{\second}$} & {$2.Messung/\si{\second}$} \\
        \midrule
        12.22   & 12.29 \\
        12.22   & 12.07 \\
        12.16   & 12.18 \\
        12.19   & 12.07 \\
        12.22   & 12.13 \\
        \bottomrule
    \end{tabular}
\end{table}


\subsection{Bestimmung der Apparaturkonstante für die große Kugel}
\subsection{Berechnung der Reynoldschen Zahl}
