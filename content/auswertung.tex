\section{Auswertung}
\label{sec:Auswertung}
\subsection{Bestimmung der Viskosität von destilliertem Wasser bei konstanter Temperatur}
Die Dichte der Kugel erhält man aus den gemessenen Werten für die Masse und den Durchmesser, welche in Tabelle aufgetragen sind.
\begin{table}[H]
  \caption{Ergebnisse der Messung.}
  \label{tab:kl_maße}
  \centering
  \sisetup{table-format=2.2}
  \begin{tabular}{cSSS}
    \toprule
    \midrule
    {$m/\si{\gram}$} & 4.45 & 4.44 & 4.46 \\
    {$D/\si{\milli\meter}$} & 15.59 & 15.60 & 15.59 \\
    \bottomrule
  \end{tabular}
\end{table}
\noindent Im Mittel erhält man dann für den Durchmesser und die Masse:
\begin{equation*}
  \bar{m}=\SI{4.45\pm 0.01}{\gram}
\end{equation*}
\begin{equation*}
  \bar{D}=\SI{15.59\pm 0.14 }{\milli\meter}
\end{equation*}
\noindent Die Ungenauigkeit errechnet sich hier über Streung des Mittelwertes:
 \begin{equation}
   \label{eq:streu}
   \Delta \bar{x} =\sqrt{\frac{1}{N(N-1)}\sum_{i=1}^N(x_i-\bar{x})^2}
 \end{equation}
 Die Kugel hat ein Volumen von:
 \begin{equation}
   V = \frac{4}{3} \pi r^3 = \SI{1.984\pm0.107e-6}{\cubic\meter}
 \end{equation}
 Die Ungenauigkeit des Volumens berechnet sich über die Gaußsche Fehlerfortpflanzung.
 Damit ergibt sich für die Kugel eine Dichte von:
 \begin{equation}
   \rho_K =\frac{\bar{m}} {V}=\SI{2242.94\pm121.07}{\kilo\gram\per\cubic\meter}
 \end{equation}
Die Unsicherheit errechnet sich hier über die Gaußsche Fehlerfortpflanzung.
Die gemessenen Werte für die Fallzeit der Kugel sind in Tabelle\ref{tab:kl_fall} eingetragen.
\begin{table}[H]
    \centering
    \caption{Fallzeiten der kleinen Kugel.}
    \label{tab:kl_fall}
    \begin{tabular}{S[table-format=2.2] S[table-format=2.2] }
        \toprule
        {$1.Messung/\si{\second}$} & {$2.Messung/\si{\second}$} \\
        \midrule
        12.22   & 12.29 \\
        12.22   & 12.07 \\
        12.16   & 12.18 \\
        12.19   & 12.07 \\
        12.22   & 12.13 \\
        \bottomrule
    \end{tabular}
\end{table}
Daraus ergibt sich im Mittel folgende Fallzeit:
\begin{equation*}
  \label{eq:fallzeit}
  \bar{t}=\SI{12.175\pm0.022}{\second}
\end{equation*}
Die Ungenauigkeit errechnet sich hier über die Streuung des Mittelwertes nach Gleichung\eqref{eq:streu}.
Die Apparaturkonstante ist in der Anleitung für die Kleine Kugel mit $K_\text{kl}=\SI{0.0764e-6}{\pascal\cubic\meter\per\kilo\gram}$\cite{v107} angegeben.
Die Dichte des Wassers wird als eine Konstante mit $\rho_\text{Fl} = 988\si{\kilo\gram\per\cubic\meter}$.
Damit folgt für die Viskosität von destilliertem Wasser folgender Wert:
\begin{equation*}
  \label{eq:vis}
  \upeta = \SI{1.167\pm9.138e-3}{\newton\second\per\meter\squared}
\end{equation*}
Die Ungenauigkeit errechnet sich hier mit der Gaußschen Fehlerfortpflanzung nach:
\begin{equation*}
  \Delta\upeta = \sqrt{(\Delta t K(\rho_K - \rho_\text{Fl})^2 +(\Delta\rho_K \cdot -K \rho_\text{Fl})^2}
\end{equation*}
\subsection{Bestimmung der Apparaturkonstante für die große Kugel}
Die Dichte der Kugel erhält man aus den gemessenen Werten für die Masse und den Durchmesser, welche in Tabelle\ref{tab:g_maße} aufgetragen sind.
\begin{table}[H]
  \caption{Ergebnisse der Messung.}
  \label{tab:g_maße}
  \centering
  \sisetup{table-format=2.2}
  \begin{tabular}{cSSS}
    \toprule
    \midrule
    {$m/\si{\gram}$} & 4.95 & 4.95 & 4.94 \\
    {$D/\si{\milli\meter}$} & 15.76 & 15.76 & 15.76 \\
    \bottomrule
  \end{tabular}
\end{table}
\noindent Im Mittel erhält man dann für den Durchmesser und die Masse:
\begin{equation*}
  \bar{m}=\SI{4.946\pm0.001}{\gram}
\end{equation*}
\begin{equation*}
  \bar{D}=\SI{15.76}{\milli\meter}
\end{equation*}
\noindent Die Ungenauigkeit errechnet sich hier über Streung des Mittelwertes nach Gleichung\eqref{eq:streu}.
Die Kugel hat ein Volumen von
\begin{equation*}
  V =\SI{2.049e-6}{\cubic\meter}
\end{equation*}
und damit eine Dichte von
\begin{equation*}
  \rho_K = \SI{2413.16\pm0.49}{\kilo\gram\per\cubic\meter}.
\end{equation*}
Die Ungenauigkeit errechnet sich über die Gaußsche Fehlerfortpflanzung:
\begin{equation*}
  \Delta \rho_K = \sqrt{{\Delta m \frac{1}{V}}^2}
\end{equation*}
Die gemessenen Werte für die Fallzeit befinden sich in Tabelle\ref{tab:fall_g}
\begin{table}[H]
    \centering
    \caption{Fallzeiten der kleinen Kugel.}
    \label{tab:fall_g}
    \begin{tabular}{S[table-format=2.2] S[table-format=2.2] }
        \toprule
        {$1.Messung/\si{\second}$} & {$2.Messung/\si{\second}$} \\
        \midrule
        70.34   & 70.13 \\
        69.56   & 69.76 \\
        70.22   & 69.46 \\
        69.22   & 69.34 \\
        69.84   & 69.69 \\
        \bottomrule
    \end{tabular}
\end{table}
 Im Mittel errechnet sich daraus ein Wert von
 \begin{equation*}
   \bar{t} = \SI{69.76\pm0.12}{\second}
 \end{equation*}
 Die Ungenauigkeit errechnet sich analog zu der Fallzeit der kleinen Kugel\eqref{eq:fallzeit}.
 Mit der vorher errechneten Vsikosität $\upeta=\SI{1.167e-3}{\newton\second\per\meter\squared}$, lässt sich die Apparaturkonstante für die große Kugel über
 \begin{equation*}
   K =\frac{\upeta}{(\rho_K - \rho_\text{Fl})\cdot \bar{t}}
 \end{equation*}
 bestimmen.
 Damit ergibt sich eine Apparaturkonstante von:
 \begin{equation*}
   K=\SI{1.1738e-8}{\pascal\cubic\meter\per\kilo\gram}
 \end{equation*}
 Die Ungenauigkeit errechnet sich über die Gaußsche Fehlerfortpflanzung nach:
 \begin{equation*}
   \Delta K = \sqrt{(\Delta t \frac{- \upeta}{(\rho_K - \rho_\text{Fl})t^2})^2 +(\Delta \upeta \frac{1}{(\rho_K - \rho_\text{Fl})t})^2 + (\Delta \rho_K \frac{-\upeta}{({\rho_K}^2 - \rho_\text{Fl})}t)^2}
 \end{equation*}
\subsection{Berechnung der Reynoldschen Zahl}
